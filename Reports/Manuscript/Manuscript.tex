% Options for packages loaded elsewhere
\PassOptionsToPackage{unicode}{hyperref}
\PassOptionsToPackage{hyphens}{url}
%
\documentclass[
]{article}
\usepackage{amsmath,amssymb}
\usepackage{lmodern}
\usepackage{iftex}
\ifPDFTeX
  \usepackage[T1]{fontenc}
  \usepackage[utf8]{inputenc}
  \usepackage{textcomp} % provide euro and other symbols
\else % if luatex or xetex
  \usepackage{unicode-math}
  \defaultfontfeatures{Scale=MatchLowercase}
  \defaultfontfeatures[\rmfamily]{Ligatures=TeX,Scale=1}
\fi
% Use upquote if available, for straight quotes in verbatim environments
\IfFileExists{upquote.sty}{\usepackage{upquote}}{}
\IfFileExists{microtype.sty}{% use microtype if available
  \usepackage[]{microtype}
  \UseMicrotypeSet[protrusion]{basicmath} % disable protrusion for tt fonts
}{}
\makeatletter
\@ifundefined{KOMAClassName}{% if non-KOMA class
  \IfFileExists{parskip.sty}{%
    \usepackage{parskip}
  }{% else
    \setlength{\parindent}{0pt}
    \setlength{\parskip}{6pt plus 2pt minus 1pt}}
}{% if KOMA class
  \KOMAoptions{parskip=half}}
\makeatother
\usepackage{xcolor}
\usepackage[margin=1in]{geometry}
\usepackage{longtable,booktabs,array}
\usepackage{calc} % for calculating minipage widths
% Correct order of tables after \paragraph or \subparagraph
\usepackage{etoolbox}
\makeatletter
\patchcmd\longtable{\par}{\if@noskipsec\mbox{}\fi\par}{}{}
\makeatother
% Allow footnotes in longtable head/foot
\IfFileExists{footnotehyper.sty}{\usepackage{footnotehyper}}{\usepackage{footnote}}
\makesavenoteenv{longtable}
\usepackage{graphicx}
\makeatletter
\def\maxwidth{\ifdim\Gin@nat@width>\linewidth\linewidth\else\Gin@nat@width\fi}
\def\maxheight{\ifdim\Gin@nat@height>\textheight\textheight\else\Gin@nat@height\fi}
\makeatother
% Scale images if necessary, so that they will not overflow the page
% margins by default, and it is still possible to overwrite the defaults
% using explicit options in \includegraphics[width, height, ...]{}
\setkeys{Gin}{width=\maxwidth,height=\maxheight,keepaspectratio}
% Set default figure placement to htbp
\makeatletter
\def\fps@figure{htbp}
\makeatother
\setlength{\emergencystretch}{3em} % prevent overfull lines
\providecommand{\tightlist}{%
  \setlength{\itemsep}{0pt}\setlength{\parskip}{0pt}}
\setcounter{secnumdepth}{5}
\newlength{\cslhangindent}
\setlength{\cslhangindent}{1.5em}
\newlength{\csllabelwidth}
\setlength{\csllabelwidth}{3em}
\newlength{\cslentryspacingunit} % times entry-spacing
\setlength{\cslentryspacingunit}{\parskip}
\newenvironment{CSLReferences}[2] % #1 hanging-ident, #2 entry spacing
 {% don't indent paragraphs
  \setlength{\parindent}{0pt}
  % turn on hanging indent if param 1 is 1
  \ifodd #1
  \let\oldpar\par
  \def\par{\hangindent=\cslhangindent\oldpar}
  \fi
  % set entry spacing
  \setlength{\parskip}{#2\cslentryspacingunit}
 }%
 {}
\usepackage{calc}
\newcommand{\CSLBlock}[1]{#1\hfill\break}
\newcommand{\CSLLeftMargin}[1]{\parbox[t]{\csllabelwidth}{#1}}
\newcommand{\CSLRightInline}[1]{\parbox[t]{\linewidth - \csllabelwidth}{#1}\break}
\newcommand{\CSLIndent}[1]{\hspace{\cslhangindent}#1}
\usepackage{float}
\let\origfigure\figure
\let\endorigfigure\endfigure
\renewenvironment{figure}[1][2] {
    \expandafter\origfigure\expandafter[H]
} {
    \endorigfigure
}
\usepackage{lscape}
\newcommand{\blandscape}{\begin{landscape}}
\newcommand{\elandscape}{\end{landscape}}
\usepackage[none]{hyphenat}
\usepackage[export]{adjustbox}
\usepackage{multirow}
\ifLuaTeX
  \usepackage{selnolig}  % disable illegal ligatures
\fi
\IfFileExists{bookmark.sty}{\usepackage{bookmark}}{\usepackage{hyperref}}
\IfFileExists{xurl.sty}{\usepackage{xurl}}{} % add URL line breaks if available
\urlstyle{same} % disable monospaced font for URLs
\hypersetup{
  pdftitle={A framework for implementing an open source and modular economic model of youth mental health that is accountable, reusable and updatable},
  pdfauthor={Matthew P Hamilton1,2,; Caroline X Gao1,3,2; Glen Wiesner4; Kate M Filia1,3; Jana M Menssink1,3; Petra Plencnerova5; David Baker1,3; Patrick D McGorry1,3; Alexandra Parker6; Jonathan Karnon7; Sue M Cotton1,3; Cathrine Mihalopoulos2},
  pdfkeywords={open source models, health economics, mental disorders},
  hidelinks,
  pdfcreator={LaTeX via pandoc}}

\title{A framework for implementing an open source and modular economic model of youth mental health that is accountable, reusable and updatable}
\author{Matthew P Hamilton\textsuperscript{1,2,*} \and Caroline X Gao\textsuperscript{1,3,2} \and Glen Wiesner\textsuperscript{4} \and Kate M Filia\textsuperscript{1,3} \and Jana M Menssink\textsuperscript{1,3} \and Petra Plencnerova\textsuperscript{5} \and David Baker\textsuperscript{1,3} \and Patrick D McGorry\textsuperscript{1,3} \and Alexandra Parker\textsuperscript{6} \and Jonathan Karnon\textsuperscript{7} \and Sue M Cotton\textsuperscript{1,3} \and Cathrine Mihalopoulos\textsuperscript{2}}
\date{}

\begin{document}
\maketitle
\begin{abstract}
\textbf{Summary: } Compared to other disciplines, health economics rarely uses modular approaches to model development that facilitate re-use of model components. Although interest in open source health economic models (OSHEMs) is growing, uptake remains limited. We propose a framework for a modular OSHEM in youth mental health. The framework includes of set of X1 good practice standards for implementing a model that is accountable (X2 standards), reusable (X3 standards) and upatable (X4 standards). We provide a rationale for each standard. The framework also includes software toolkit of six R packages for authoring model modules (data structures and algorithms), supplying those modules with data and using the modules to implement reproducible modelling analyses. We describe an early application of the framework to developing utility mapping models that meet X5 standards. We discuss the potential benefits and challenges of extending this initial work to develop a more extensive model for undertaking economic analyses in youth mental health. \newline \newline \textbf{Code: } Visit \url{https://www.ready4-dev.com} for more information about how to find, install and apply the framework and model. \newline \newline
\end{abstract}

\textsuperscript{1} Orygen, Parkville, Australia\\
\textsuperscript{2} School of Public Health and Preventive Medicine, Monash University, Clayton, Australia\\
\textsuperscript{3} Centre for Youth Mental Health; The University of Melbourne, Parkville, Australia\\
\textsuperscript{4} Heart Foundation, Melbourne, Australia\\
\textsuperscript{5} headspace National Youth Mental Health Foundation, Melbourne, Australia\\
\textsuperscript{6} Victoria University, Footscray, Australia\\
\textsuperscript{7} Flinders University, Adelaide, Australia

\textsuperscript{*} Correspondence: \href{mailto:matthew.hamilton@orygen.org.au}{Matthew P Hamilton \textless{}\href{mailto:matthew.hamilton@orygen.org.au}{\nolinkurl{matthew.hamilton@orygen.org.au}}\textgreater{}}

\hypertarget{introduction}{%
\section{Introduction}\label{introduction}}

Computational models, particularly those addressing economic topics, have become essential tools for health policy development {[}1,2{]}. Although influential and widely used, health economic models typically have a number of limitations that at best restrict their usefulness and which in some cases may facilitate harmful misuse.

There is a strong case for making health economists more accountable for the appropriate use and social acceptability of their models. Users of models should be able to assess their adequacy for a particular purpose {[}3{]}. This goal is difficult to achieve in the current context of the poor reproducibility {[}4--6{]} and frequently insufficient validation {[}7{]} of health economic models. Value judgments strongly shape health economic analyses, yet are rarely made explicit, omissions that may lead to socially unacceptable policy recommendations {[}8{]}. A modelling team's value judgments about what questions to address, the most important features of a system to represent and the weighting of different types of evidence may be poorly aligned with those of the people impacted by decisions informed by model analyses {[}9{]}. As health economic models adopt more sophisticated techniques, the need for accountability grows. More complex models may be more prone to propagation errors {[}10{]} and models designed to address multiple questions should be expected to meet more onerous verification and validation obligations {[}11,12{]}. The nature and extent of individual model authorship contributions may be less clear in models implemented over longer time-frames with a large and changing group of collaborators {[}\textbf{thompson2022?}{]}.

Collaborative approaches that encourage re-use of models by other modellers can help make modelling pojects more tractable {[}13{]}. However, as many health economic models are owned by pharmaceutical companies and consultancies, commercial considerations can limit the reuse of models and their constituent code and data {[}12{]}. Not all data included in health economic models can be ethically made publicly available due to privacy and confidentiality requirements. Transferring a health economic model developed for one jurisdiction for application in another will involve retaining some features and updating others {[}14{]}. Use of concepts with standardised meanings across jurisdictions {[}15{]}, supplying context specific data as replaceable data-packs and distributing source code under licenses that allow derivative works are all steps that can enhance model transferability.

Health economic models should be updated and refined as new evidence emerges and decision contexts change {[}16{]}, but this occurs infrequently {[}17{]}. Decision makers are frequently left to rely on economic analyses that become less valid and relevant each year. A model can grow stale when changes in the real world, whether sudden (e.g.~new pandemics) or gradual (e.g.~demographic), is not represented by corresponding updates to relevant model features. Even when the system being modelled remains relatively stable, new research findings may potentially invalidate aspects of a model's original conceptualisation and implementation. Funding for health economic modelling projects rarely extend to provision of medium term support for model updates and improvements. The career trajectories of health economists can also mitigate against adequate maintenance of a model, it being relatively common for model authors to have moved on from the team that owns the model and / or from working on the health condition for which the model was developed.

A potential strategy for improving model accountability, reusability and updatability is to make models both modular and open-source.

A modular approach to model building involves models being constructed by combining multiple replaceable sub-models (modules) {[}18{]}. When used as model sub-components, modules communicate with other modules but they can be also run independently {[}19{]}. Advantages of modular models include feasibility (large projects are broken into smaller tasks, with each component independently developed and tested) and flexibility (making it easier to selectively replace or update specific parts of a model and to scale up or down the level of granularity) {[}18{]}. Modular approaches are currently being used to facilitate the development of complex computational models in disciplines such as biology {[}18{]}, neuroscience {[}20{]} and ecology {[}19{]}. Although the related and enabling concept of reusable reference models has been recommended in health economics {[}21{]}, peer reviewed studies describing modular health economic models remain relatively rare, though examples exist in tuberculosis {[}22{]} and cardiology {[}23{]}.

The benefits of modular models can be enhanced if multiple modelling teams have the capacity to contribute and use model modules. To facilitate this, modular models may be implemented as open source projects that grant liberal permissions to access and re-use model source code and data {[}18--20{]}. Although there appears to be in principle support from many health economists for greater use of open source health economic models (OSHEMs) {[}24{]}, actual implementations are rare {[}12,25,26{]}. Barriers to adoption of OSHEMs include concerns about intellectual property, confidentiality, model misuse and the resources required to support open source implementations {[}24,27{]}. Adherence to good practice guidance is an essential requirement for healthcare modelling {[}2{]}, but guidelines for implementing OSHEMs remain scarce, piecemeal and need improving {[}28{]}.

We plan to develop a modular OSHEM of youth mental health. Open source approaches have been recommended to help develop the mental health modelling field {[}29{]} but only one mental health related model (in Alcohol Use Disorder {[}30{]}) is currently indexed {[}RECHECK{]} in the Open Source Models Clearinghouse {[}25,31{]}. Other than our own work, we are aware of just one open source mental health reference model - in Major Depressive Disorder - that is currently in development {[}32{]}.

In this paper, we introduce a framework for developing a modular youth mental health OSHEM. Specifically we describe the framework's:

\begin{itemize}
\item
  good practice standards for implementing a modular OSHEM;
\item
  software toolkit for developing a modular OSHEM consistent with those good practice standards; and
\item
  application to implementing model modules to support the development and use of utility mapping models.
\end{itemize}

\hypertarget{guidelines-for-oshems}{%
\section{Guidelines for OSHEMs}\label{guidelines-for-oshems}}

Guidelines on health economic model transparency were published ten years ago {[}11{]} and made recommendations on documenting models; however, notably missing were recommendations on the sharing of model code and data. More recent and more general modelling guidance {[}2{]} recommends the sharing of code and data through digital repository services and the use of version control systems. Based on these and other existing sources of guidance and our own experience, we propose 20 guidelines for implementing OSHEMs that are \textbf{TIMELY} (Transparent, Iterative, Modular, Epitomising, Licensed and Yielding - see Table @ref(tab:timelygls)).

\begin{table}

\caption{(\#tab:timelygls)TIMELY Guidelines for OSHEMs}
\centering
\begin{tabular}[t]{>{}ll>{\raggedright\arraybackslash}p{35em}}
\toprule
Characteristic & Guideline & Meaning\\
\midrule
\addlinespace[0.3em]
\multicolumn{3}{l}{\textbf{Transparent - people can easily see how a model has been implemented and tested}}\\
\hspace{1em}\textbf{} & T1 & Uniquely identified copies of model code and data are permanently archived in open online repositories\\
\hspace{1em}\textbf{} & T2 & Model code and data are documented\\
\hspace{1em}\textbf{} & T3 & Model code uses a simple and consistent syntax\\
\hspace{1em}\textbf{} & T4 & Model analyses and reporting are implemented using literate programming\\
\hspace{1em}\textbf{} & T5 & Code coverage is reported\\
\hspace{1em}\textbf{} & T6 & All parts of a study analysis and reporting workflow can be reproduced and/or replicated\\
\addlinespace[0.3em]
\multicolumn{3}{l}{\textbf{Iterative - a model is routinely updated to maintain and improve validity}}\\
\hspace{1em}\textbf{} & I1 & Model code and data are version controlled\\
\hspace{1em}\textbf{} & I2 & Model code and data use semantic versioning\\
\hspace{1em}\textbf{} & I3 & Continuous integration is used to verify model code updates\\
\hspace{1em}\textbf{} & I4 & Deprecation conventions are used to retire model code and data\\
\addlinespace[0.3em]
\multicolumn{3}{l}{\textbf{Modular - models and their components can be combined to extend their scope}}\\
\hspace{1em}\textbf{} & M1 & Model code and data are stored and managed separately\\
\hspace{1em}\textbf{} & M2 & Model code defines encapsulating data structures\\
\addlinespace[0.3em]
\multicolumn{3}{l}{\textbf{Epitomising - model code can be re-used in other decision contexts}}\\
\hspace{1em}\textbf{} & E1 & Model code is distributed as libraries of classes and functions\\
\hspace{1em}\textbf{} & E2 & Model code defines inheriting data-structures\\
\hspace{1em}\textbf{} & E3 & Test data is available to demonstrate generalised applications of model code\\
\addlinespace[0.3em]
\multicolumn{3}{l}{\textbf{Licensed - a model and its components are persistently re-usable by other modellers}}\\
\hspace{1em}\textbf{} & L1 & Model code is made available for re-use under copyleft or permissive licenses\\
\hspace{1em}\textbf{} & L2 & Non-confidential model data is licensed for liberal re-use (subject to additional terms for de-identified human data)\\
\hspace{1em}\textbf{} & L3 & Model code and data are distributed with tools to support appropriate citation\\
\addlinespace[0.3em]
\multicolumn{3}{l}{\textbf{Yielding - a model can be simply, flexibly and reliably used as a decision aid}}\\
\hspace{1em}\textbf{} & Y1 & Models are distributed with validated tools to support their safe and appropriate re-use\\
\hspace{1em}\textbf{} & Y2 & Simple user-interfaces allow non-technical users to configure and run models\\
\bottomrule
\end{tabular}
\end{table}

\hypertarget{transparent-models}{%
\subsection{Transparent models}\label{transparent-models}}

A range of tools and practices are available to help make models more transparent. The most efficient way to widely disseminate code and data may be to use existing open science infrastructure {[}2{]} to make permanently accessible and uniquely identfied digital artefacts. Model code and data also need to be clearly documented, potentially with different versions for technical and non-technical users {[}11{]}. Consistent use of meaningful naming conventions when authoring code is recommended {[}33,34{]}. Code can be made easier to follow by using the practices of abstraction {[}35{]}, where only simple, high level commands are routinely exposed to reviewers, and polymorphism {[}36{]}, where the same command (e.g.~``simulate'') can be reused to implement different algorithms of the same type. Programs to implement model analyses can be made comprehensible to even non-technical users through the use of literate programming techniques to render documents that integrate computer code with plain English descriptions.

An essential component of quality assuring health economic models is verification - ensuring that calculations are correct and consistent with model specifications {[}37{]}. One useful concept for informing model users about the extensiveness of verification checks is code coverage {[}38{]} - the proportion of model code that has been explicitly tested. Finally, transcription errors - mistakes introduced when transferring data between sources, models and reports - are very common in health economic models {[}39{]}. The risk of these errors might be lower if there was full transparency across all steps in a study workflow. Scientific computing tools now make it relatively straightforward to author programs that reproducibly execute all steps in data ingest, processing and reporting {[}33{]}.

\hypertarget{iterative-models}{%
\subsection{Iterative models}\label{iterative-models}}

To avoid OSHEMs going stale - losing validity and usefulness with time - they should be routinely updated. Each update of code and data should be uniquely identifiable and retrievable, a goal that can be facilitated by use of version control tools {[}2{]}.Labeling each change using semantic versioning {[}40{]} conventions can signal the potential importance of an update to users of model code and data. Continuous integration {[}41{]} tools can help verify that each code update passes multiple quality tests.
Finally, using deprecation conventions that take an informative and staged approach to retiring old code and data reduces the risk that model revisions have unintended consequences on third party users.\\

\hypertarget{modular-models}{%
\subsection{Modular models}\label{modular-models}}

An important consideration when combining model components (or modules) is to ensure that interactions between two modules do not compromise the validity of either. Using the coding practice of encapsulation {[}35{]} can help ensure that model modules can be safely combined {[}42{]}.

\hypertarget{epitomising-models}{%
\subsection{Epitomising models}\label{epitomising-models}}

A key challenge to generalising health economic models is that they are typically developed to inform a decision problem with a highly specific jurisdictional context. However, a number of choices about how these models are implemented can significantly increase the re-usability of model code in other contexts.

Writing code as collections of functions (short, self-contained and reusable algorithms that each perform a discrete task) is recommended as good practice for scientific computing {[}33{]}. When bundled for distribution as libraries, functions have the potential to be widely re-used, often in contexts very different than those they were originally developed for. A special type of function, called a method, can only be applied to a pre-defined class of data structure. Due to the coding concept of inheritance {[}35{]}, the more restricted nature of methods can be used to enhance the re-usability of model code in different decision contexts {[}42{]}. For example, when generalising a model developed for the Australian context to a UK context, one could create a class that initially inherits all of the methods defined for the Australian model and then write new or replacement methods as needed for the UK model.Whatever type of functions are written for a modelling project, it is good practice to make available test or toy data to demonstrate their use {[}33{]}.

\hypertarget{licensed-models}{%
\subsection{Licensed models}\label{licensed-models}}

To make model code and data widely re-usable by others, it is important to provide users with appropriate and explicit permissions. In the context of open source models, there are two broad categories of licensing options. Some guidance strongly recommends the use of permissive licensing {[}33{]} that provides users with great flexibility as to the purposes (including commercial) for which the content could be re-used. An alternative approach is to use copyleft licenses {[}43{]} that can require content users to distribute any derivative works they create under similar open source arrangements.The most suitable approach may differ for code and data.

For code, it may be appropriate to adopt the prevailing open source licensing practice within the programming language being used. For data, it may not be sufficient to simply choose between a permissive license like the Public Domain Dedication (CC0) {[}44{]} or a copyleft option such as the Attribution-Share Alike (CC-BY-SA) {[}45{]}. Responsible custodianship of some de-identified mental health data may involve using or adapting template terms of use {[}46{]} which have a number of ethical clauses (for example, prohibiting efforts to re-identify research participants). Licenses may or may not specify that model re-users must give appropriate acknowledgement to model authors. Citation tools can be distributed with each individual code or data item to inform re-users of the desired attribution.

\hypertarget{yielding-models}{%
\subsection{Yielding models}\label{yielding-models}}

OSHEMs can be time and skills intensive for modellers to develop - but they should be easy for others to use. Statistical models are a common output of health economic evaluations, but they are often not reported in a format that enables others to confidently and reliably re-use them {[}47{]}. Open source approaches can help address this by disseminating code artefacts that enable easy and appropriate use of a statistical model to make predictions with new data. However, great care must be exercised when doing so if models are derived from data on human subjects as some software artefacts by default contain a copy of the source dataset. Such dataset copies must therefore be replaced (for example, with synthetic data) and the amended artefact's predictive performance then retested before any public release. Another way to make OSHEMs easier to use is to develop simple user-interfaces for non-technical users.

\hypertarget{modelling-toolkit}{%
\section{Modelling Toolkit}\label{modelling-toolkit}}

We have developed a toolkit to help streamline the process of developing OSHEMs that meet the TIMELY standards. The toolkit is comprised of online repositories and software.

\hypertarget{repositories}{%
\subsection{\texorpdfstring{Repositories }{Repositories }}\label{repositories}}

We created a GitHub organisation where all framework software source code is stored, documented, version controlled and continuously integrated {[}48{]}. To store citable archived copies of release copies of our software, we created a Zenodo community {[}49{]}. Finally, to manage datasets for use in models developed with the framework, we created a dedicated collection within the Harvard Dataverse {[}50{]}.

\hypertarget{software}{%
\subsection{\texorpdfstring{Software }{Software }}\label{software}}

A coding framework for OSHEMs developed in the language R includes standardised approaches to directory structure and naming conventions {[}34{]}.

As a foundation for implementing the framework, we authored a development version R package that defines a novel syntax and a template class for model modules {[}51{]}. To enable the syntax and module template be applied to modelling projects, we then created five additional development version R packages that provide tools for authoring:

\begin{itemize}
\item
  documented model modules {[}52{]};
\item
  documented functions (including methods), written in a consistent house style {[}53{]};
\item
  citable, quality assured R packages {[}54{]};
\item
  model datasets {[}55{]};
\item
  model analyses and reports {[}56{]}.
\end{itemize}

The six R packages, their primary focus, the TIMELY standards they support and the third-party packages they depend on are summarised in Table @ref(tab:cpkgs). When used in conjunction with framework repositories, the six packages extend existing R packages provide strong support for implementing 16 of the TIMELY standards. However, the software only weakly supports implementing the standards relating to disseminating statistical models (Y1) and user-interface development (Y2) and does not yet provide any workflow tools to help implement the standards for code coverage (T5) and deprecation conventions (I4). Standards not supported or weakly supported by our software can be met with existing developer tools in R and we plan to progressively integrate these third-party tools with our own in future releases of our software. Another future priority is to submit production versions of each R package for review by and archiving on the Comprehensive R Archive Network (CRAN) {[}57{]}.

\begin{table}

\caption{(\#tab:cpkgs)Framework software to enable the readyforwhatsnext model to meet TIMELY guidelines}
\centering
\begin{tabular}[t]{ll>{}l>{\raggedright\arraybackslash}m{24em}}
\toprule
Package & Focus & Guideline      & Depends on these R packages\\
\midrule
\includegraphics[valign=M,width=1.9cm,raise=4mm]{../Data/images/ready4_logo.png} & Syntax & T3           M3 & assertthat bib2df dataverse dplyr fs Hmisc kableExtra knitr lifecycle magrittr methods natmanager piggyback purrr readr readxl rlang rmarkdown rvest stats stringi stringr testit testthat tibble tidyRSS tools utils zen4R\\
\includegraphics[valign=M,width=1.9cm,raise=4mm]{../Data/images/ready4fun_logo.png} & Functions & T2-3             E1 & desc devtools dplyr generics gert Hmisc knitr lifecycle lubridate magrittr methods piggyback pkgdown purrr readxl ready4 ready4show ready4use rlang sinew stats stringi stringr testit testthat tibble tidyr tools usethis utils xfun\\
\includegraphics[valign=M,width=1.9cm,raise=4mm]{../Data/images/ready4class_logo.png} & Modules & T2-3         M2   E1-2 & devtools dplyr fs gtools Hmisc knitr lifecycle magrittr methods purrr ready4 ready4fun ready4show rlang stats stringi stringr testit testthat tibble tidyr usethis utils\\
\includegraphics[valign=M,width=1.9cm,raise=4mm]{../Data/images/ready4pack_logo.png} & Libraries & T1      I1-3  M1-2 E1   L1,3 & dataverse dplyr knitr lifecycle magrittr methods purrr ready4 ready4class ready4fun rlang stringr testthat tibble tidyr utils\\
\includegraphics[valign=M,width=1.9cm,raise=4mm]{../Data/images/ready4use_logo.png} & Data & T1-2         M1   E3   L2-3 & data.table dataverse dplyr fs Hmisc knitr lifecycle magrittr methods piggyback purrr readxl ready4 ready4show rlang stats stringi stringr testit testthat tibble tidyr utils\\
\addlinespace
\includegraphics[valign=M,width=1.9cm,raise=4mm]{../Data/images/ready4show_logo.png} & Reporting & T2,4,6                           Y1-2 & dataverse DescTools dplyr flextable grDevices here Hmisc kableExtra knitr knitrBootstrap lifecycle magrittr methods officer purrr ready4 rlang rmarkdown stringi stringr testthat tibble tidyr utils xtable\\
\bottomrule
\end{tabular}
\end{table}

\blandscape

\begin{longtable}[]{@{}
  >{\raggedright\arraybackslash}p{(\columnwidth - 12\tabcolsep) * \real{0.0040}}
  >{\raggedright\arraybackslash}p{(\columnwidth - 12\tabcolsep) * \real{0.0094}}
  >{\raggedright\arraybackslash}p{(\columnwidth - 12\tabcolsep) * \real{0.0335}}
  >{\raggedright\arraybackslash}p{(\columnwidth - 12\tabcolsep) * \real{0.0375}}
  >{\raggedright\arraybackslash}p{(\columnwidth - 12\tabcolsep) * \real{0.0067}}
  >{\raggedright\arraybackslash}p{(\columnwidth - 12\tabcolsep) * \real{0.0094}}
  >{\raggedright\arraybackslash}p{(\columnwidth - 12\tabcolsep) * \real{0.8995}}@{}}
\caption{(\#tab:checktb)TIMELY Checklist applied to utility mapping study}\tabularnewline
\toprule()
\begin{minipage}[b]{\linewidth}\raggedright
\end{minipage} & \begin{minipage}[b]{\linewidth}\raggedright
\end{minipage} & \begin{minipage}[b]{\linewidth}\raggedright
Standard
\end{minipage} & \begin{minipage}[b]{\linewidth}\raggedright
\end{minipage} & \begin{minipage}[b]{\linewidth}\raggedright
Met?
\end{minipage} & \begin{minipage}[b]{\linewidth}\raggedright
\end{minipage} & \begin{minipage}[b]{\linewidth}\raggedright
Description
\end{minipage} \\
\midrule()
\endfirsthead
\toprule()
\begin{minipage}[b]{\linewidth}\raggedright
\end{minipage} & \begin{minipage}[b]{\linewidth}\raggedright
\end{minipage} & \begin{minipage}[b]{\linewidth}\raggedright
Standard
\end{minipage} & \begin{minipage}[b]{\linewidth}\raggedright
\end{minipage} & \begin{minipage}[b]{\linewidth}\raggedright
Met?
\end{minipage} & \begin{minipage}[b]{\linewidth}\raggedright
\end{minipage} & \begin{minipage}[b]{\linewidth}\raggedright
Description
\end{minipage} \\
\midrule()
\endhead
T1 & ~ & Publicly archived & ~ ~ ~ ~ & Yes & ~ & 5 libraries (youthvars, scorz, specific, TTU and youthu) {[}58--62{]}, 3 programs {[}63--65{]}, 2 sub-routines {[}66,67{]} and 2 datasets {[}68,69{]} are permanently archived with unique identifiers. \\
T2 & ~ & Documented & ~ ~ ~ ~ & Yes & ~ & All code libraries have documenting websites with URLs that concatenate `\url{https://ready4-dev.github.io/}' and the package name (e.g.~\url{https://ready4-dev.github.io/youthvars}). All three Markdown programs are self-documenting, with one {[}63{]} including additional instructions in a README file. Only one sub-routine {[}67{]} is documented with a meaningful README file. All datasets have meaningful metadata descriptors. \\
T3 & ~ & Consistent syntax & ~ ~ ~ ~ & Yes & ~ & All libraries, programs and sub-routines use the same house style, which allows most library documentation to be written by algorithms from the ready4fun package {[}53{]}. All libraries except {[}62{]} use framework syntax, as does one program {[}63{]}. \\
T4 & ~ & Literately programmed & ~ ~ ~ ~ & Yes & ~ & All programs use literate programming. \\
T5 & ~ & Code coverage & ~ ~ ~ ~ & No & ~ & No current reporting of code coverage. \\
T6 & ~ & Reproducible & ~ ~ ~ ~ & Yes & ~ & All parts of the study workflow from raw data ingest through to data processing, analysis, reporting and dissemination of study outputs can be reproduced (if granted access to source data) or replicated (using supplied synthetic data) with one program {[}63{]}. \\
I1 & ~ & Version controlled & ~ ~ ~ ~ & Yes & ~ & All code is version controlled using Git and GitHub. All source code is available in a GitHub organisation {[}48{]}. \\
I2 & ~ & Semanticly versioned & ~ ~ ~ ~ & Yes & ~ & Semantic version is used in all code. As no code library has yet been submitted to CRAN, only the development version extensions of each version number have been incremented to date. \\
I3 & ~ & Continuously integrated & ~ ~ ~ ~ & Yes & ~ & All six libraries use continuous integration (CI). CI results for each library can be viewed at a URL that concatenates `\url{https://github.com/ready4-dev/}', the package name and `/actions' (e.g.~\url{https://github.com/ready4-dev/youthvars/actions}) \\
I4 & ~ & Deprecation & ~ ~ ~ ~ & Yes & ~ & Retired code is deprecated using lifecycle package tools (e.g.~everything after ``\#\# DEPRECATED FNS'' in \url{https://github.com/ready4-dev/youthvars/blob/main/data-raw/fns/add.R} ). Package vignettes and datasets are also deprecated e.g.~\url{https://ready4-dev.github.io/youthvars/articles/Replication_DS.html} ) \\
M1 & ~ & Separate code and data & ~ ~ ~ ~ & Yes & ~ & All development code is stored on repos in a GitHub organisation {[}48{]} and all archived releases are available in a Zenodo community {[}49{]}. All non-confidential data is stored in repositories within a Harvard Dataverse collection {[}50{]}. \\
M2 & ~ & Encapsulated & ~ ~ ~ ~ & Yes & ~ & Four {[}58--61{]} out of five libraries include encapsulated modules. Examples are the items beginning with Scorz, Specific and Youthvars that are listed in this table: \url{https://ready4-dev.github.io/ready4/articles/V_01.html\#current-ready4-framework-modules} as well as the S4 classess from the TTU package listed here: \url{https://ready4-dev.github.io/TTU/reference/index.html} \\
E1 & ~ & Libraries & ~ ~ ~ ~ & Yes & ~ & All code libraries include functions. The most complete list of functions for each library is available by clicking the `Manual - Developer (PDF)' link on each package's documentation homepage (see item T2 above). \\
E2 & ~ & Inheriting & ~ ~ ~ ~ & Yes & ~ & All modules (see item M2) inherit from the Ready4Module class and can be inherited from. \\
E3 & ~ & Test data & ~ ~ ~ ~ & Yes & ~ & Two synthetic datasets and their data dictionaries are publicly available in a data repository {[}69{]}. One (ymh\_clinical\_tb.RDS) closely resembles the study dataset and was released so that the main study algorithm {[}63{]} can be rerun by those without access to the confidential study dataset. The other (eq5d\_ds\_dict.RDS) is deliberately different to the source dataset in both variable naming convention and the concepts used for predictors and outcome measures. It was created to demonstrate generalised applications of study algorithms. \\
L1 & ~ & Copyleft code & ~ ~ ~ ~ & Yes & ~ & All code libraries, programs and sub-routines use GPL-3 licenses. \\
L2 & ~ & Liberal, safe data terms & ~ ~ ~ ~ & Yes & ~ & Datasets use amended version of template provided by Harvard Dataverse {[}46{]}. \\
L3 & ~ & Citation tools & ~ ~ ~ ~ & Yes & ~ & All libraries have a CITATION file in the inst directory. All code repositories have a CITATION.cff file. All datasets have citation generating metadata. \\
Y1 & ~ & Prediction tools & ~ ~ ~ ~ & Yes & ~ & Model catalogues (PDF files beginning with `AAA\_TTU\_MDL\_CTG') are available in the study results dataset {[}68{]} and describe the predictive performance of all models under a variety of usage regimes (including when the source dataset in the R model object is replaced with fake data). The youthu library {[}62{]} includes tools for searching for and applying models compatible with different types of input data. An example program to demonstrate this functionality is available in both RMarkdown {[}64{]} and rendered PDF formats (the `Application.pdf' file in the study results dataset {[}68{]}). \\
Y2 & ~ & User interface & ~ ~ ~ ~ & No & ~ & No Shiny app user interface has yet been developed. \\
\bottomrule()
\end{longtable}

\elandscape

\hypertarget{application}{%
\section{Application}\label{application}}

\hypertarget{readyforwhatsnext}{%
\subsection{readyforwhatsnext}\label{readyforwhatsnext}}

Our approach to model development is to undertake a number of discrete modelling projects of the people, places, platforms and programs that shape the mental health and wellbeing of young people and to progressively link them together by means of a common framework. To model people we are developing synthetic representations of populations of interest {[}69{]} that describe relevant individual characteristics and their relationships, algorithms that map psychological measures to health utility {[}70{]} and choice models for predicting the helpseeking behaviour of young people {[}71{]}. Our in-development model of places {[}72{]} has the aim of synthesising geometry and spatial attribute data to characterise the geographic distribution of relevant demographic, environmental, epidemiological and service infrastructure features. We are in the early stages of a multi-annual project to develop a service platform model that will represent the processes and operations of a complex primary youth mental health service. We also plan to extend and update our prior work reviewing economic evidence relating to youth mental health programs {[}73{]} so that it can be integrated with this model.

Our initial work on \textbf{readyforwhatsnext} is focused on Victoria, Australia but the framework we are using to develop it is designed to facilitate extension by ourselves and others to different jurisdictional decision contexts. Progress is reported on a project website {[}74{]}.

\hypertarget{utility-mapping}{%
\subsection{Utility mapping}\label{utility-mapping}}

We used framework toolkits to develop model code and datasets for implementing a utility mapping study that has previously been described {[}70{]}. Table @ref(tab:checktb) assesses that study against each TIMELY standard, 18 of which the study was able to meet. The two current standards where the study falls short are in reporting code coverage and including a user-interface. Both these items are current development priorities for our team.

\hypertarget{discussion}{%
\section{Discussion}\label{discussion}}

Mental disorders impose high health, social and economic burdens worldwide {[}75,76{]}. Much of this burden is potentially avertable {[}77{]}, but poorly financed and organised mental health systems are ill-equipped for this challenge{[}78,79{]}. The large and widespread additional mental health burdens recently observed during the COVID-19 pandemic{[}80{]} and predicted as a potential future consequence of climate change {[}81{]}, highlight the need to improve the resilience and adaptability of these systems. To help stem growing demand for mental health services, policymakers have also been encouraged to place greater emphasis on tackling the social determinants of mental disorders {[}82{]}.

Major mental health reform programs, can involve the identification, prioritisation, sequencing, targeting and monitoring of multiple interdependent initiatives.

The significant deficits in our understanding of the systems in which mental disorders emerge and are treated {[}83{]} suggest that there is ample scope for mental health systems models to progressively improve their validity over time. Prospective work could address the weak theoretical underpinnings for understanding complex mental health systems {[}84{]}. For example, it remains unclear why increased investments in mental health care have yet to discernibly reduce the prevalence and burden of mental disorders{[}85{]}. The literature, and evidence base, regarding how the requirements, characteristics and performance of mental health services are shaped by spatiotemporal context needs to be further developed {[}86{]}. There is also a need for better evidence to identify the social determinants of mental disorders most amenable to preventative interventions, and for which population sub-groups such interventions would be most effective {[}87{]}.

The development, validation and updating of more complex mental health economic models implemented over longer time frames may be too onerous a burden for a single modelling team.

Developing networks of modellers working on common health conditions has been recommended as a strategy for improving model validity {[}28{]} and some of us are part of a nascent initiate of this type in mental health {[}88{]}.
Similarly, developing partnerships between modellers and decision-makers across the life-cycle of a modelling project can help ensure models are appropriately conceptualised, implemented and have practical utility as decision aids {[}89,90{]}.

\hypertarget{conclusion}{%
\section{Conclusion}\label{conclusion}}

We have identified a number of standards that we believe are appropriate to implementing quality OSHEMs in youth mental health. Most of these standards are probably relevant to OSHEMs in other health conditions, though some such as the copyleft licensing may be less relevant to modellers using different tools. Our framework toolkits can help support standardised approaches to dynamic systems model development that are important for collaborative and interdependent projects.\\

\hypertarget{acknowledgement}{%
\subsection{Acknowledgement}\label{acknowledgement}}

The authors would like to acknowledge the contribution of John Gillam who provided advisory input to this research.

\hypertarget{availability-of-data-and-materials}{%
\subsection*{Availability of data and materials}\label{availability-of-data-and-materials}}
\addcontentsline{toc}{subsection}{Availability of data and materials}

Development versions of all code repositories referenced in this article are available in \url{https://github.com/ready4-dev/} . Archived code releases are available in \url{https://zenodo.org/communities/ready4} .

All data repositories referenced in this article are available in \url{https://dataverse.harvard.edu/dataverse/ready4} .

\hypertarget{ethics-approval}{%
\subsection*{Ethics approval}\label{ethics-approval}}
\addcontentsline{toc}{subsection}{Ethics approval}

Framework development did not involve human subject research and was not ethically reviewed. The worked example of framework application is a previously reported study that was reviewed and granted approval by the University of Melbourne's Human Research Ethics Committee, and the local Human Ethics and Advisory Group (1645367.1).

\hypertarget{funding}{%
\subsection*{Funding}\label{funding}}
\addcontentsline{toc}{subsection}{Funding}

Framework development was funded by Orygen, VicHealth and Victoria University. The previously reported study used as a worked example was funded by the National Health and Medical Research Council (NHMRC, APP1076940), Orygen and headspace.

\hypertarget{conflict-of-interest}{%
\subsection*{Conflict of Interest}\label{conflict-of-interest}}
\addcontentsline{toc}{subsection}{Conflict of Interest}

None declared.

\newpage

\hypertarget{references}{%
\section*{References}\label{references}}
\addcontentsline{toc}{section}{References}

\hypertarget{refs}{}
\begin{CSLReferences}{0}{0}
\leavevmode\vadjust pre{\hypertarget{ref-dakin2015influence}{}}%
\CSLLeftMargin{1. }%
\CSLRightInline{Dakin H, Devlin N, Feng Y, Rice N, O'Neill P, Parkin D. The influence of cost-effectiveness and other factors on nice decisions. Health economics. Wiley Online Library; 2015;24: 1256--1271. }

\leavevmode\vadjust pre{\hypertarget{ref-Erdemir2020}{}}%
\CSLLeftMargin{2. }%
\CSLRightInline{Erdemir A, Mulugeta L, Ku JP, Drach A, Horner M, Morrison TM, et al. Credible practice of modeling and simulation in healthcare: Ten rules from a multidisciplinary perspective. Journal of translational medicine. 2020;18: 369. doi:\href{https://doi.org/10.1186/s12967-020-02540-4}{10.1186/s12967-020-02540-4}}

\leavevmode\vadjust pre{\hypertarget{ref-thompson2019escape}{}}%
\CSLLeftMargin{3. }%
\CSLRightInline{Thompson EL, Smith LA. Escape from model-land. Economics. De Gruyter Open Access; 2019;13. }

\leavevmode\vadjust pre{\hypertarget{ref-Jalali2021}{}}%
\CSLLeftMargin{4. }%
\CSLRightInline{Jalali MS, DiGennaro C, Guitar A, Lew K, Rahmandad H. Evolution and reproducibility of simulation modeling in epidemiology and health policy over half a century. Epidemiologic Reviews. 2021;43: 166--175. doi:\href{https://doi.org/10.1093/epirev/mxab006}{10.1093/epirev/mxab006}}

\leavevmode\vadjust pre{\hypertarget{ref-McManus2019}{}}%
\CSLLeftMargin{5. }%
\CSLRightInline{McManus E, Turner D, Sach T. Can you repeat that? Exploring the definition of a successful model replication in health economics. Pharmacoeconomics. 2019;37: 1371--1381. doi:\href{https://doi.org/10.1007/s40273-019-00836-y}{10.1007/s40273-019-00836-y}}

\leavevmode\vadjust pre{\hypertarget{ref-Bermejo2017}{}}%
\CSLLeftMargin{6. }%
\CSLRightInline{Bermejo I, Tappenden P, Youn J-H. Replicating health economic models: Firm foundations or a house of cards? PharmacoEconomics. 2017;35: 1113--1121. doi:\href{https://doi.org/10.1007/s40273-017-0553-x}{10.1007/s40273-017-0553-x}}

\leavevmode\vadjust pre{\hypertarget{ref-Ghabri2019}{}}%
\CSLLeftMargin{7. }%
\CSLRightInline{Ghabri S, Stevenson M, Möller J, Caro JJ. Trusting the results of model-based economic analyses: Is there a pragmatic validation solution? Pharmacoeconomics. 2019;37: 1--6. doi:\href{https://doi.org/10.1007/s40273-018-0711-9}{10.1007/s40273-018-0711-9}}

\leavevmode\vadjust pre{\hypertarget{ref-duckett2022journey}{}}%
\CSLLeftMargin{8. }%
\CSLRightInline{Duckett S. A journey towards a theology of health economics and healthcare funding. Theology. SAGE Publications Sage UK: London, England; 2022;125: 326--334. }

\leavevmode\vadjust pre{\hypertarget{ref-thompson2022escape}{}}%
\CSLLeftMargin{9. }%
\CSLRightInline{Thompson E. Escape from model land: How mathematical models can lead us astray and what we can do about it. New Yourk: Basic Books; 2022. }

\leavevmode\vadjust pre{\hypertarget{ref-Saltelli2019}{}}%
\CSLLeftMargin{10. }%
\CSLRightInline{Saltelli A. A short comment on statistical versus mathematical modelling. Nature Communications. 2019;10: 3870. doi:\href{https://doi.org/10.1038/s41467-019-11865-8}{10.1038/s41467-019-11865-8}}

\leavevmode\vadjust pre{\hypertarget{ref-Eddy2012}{}}%
\CSLLeftMargin{11. }%
\CSLRightInline{Eddy DM, Hollingworth W, Caro JJ, Tsevat J, McDonald KM, Wong JB. Model transparency and validation: A report of the ISPOR-SMDM modeling good research practices task force-7. Med Decis Making. 2012;32: 733--43. doi:\href{https://doi.org/10.1177/0272989x12454579}{10.1177/0272989x12454579}}

\leavevmode\vadjust pre{\hypertarget{ref-Feenstra2022}{}}%
\CSLLeftMargin{12. }%
\CSLRightInline{Feenstra T, Corro-Ramos I, Hamerlijnck D, Voorn G van, Ghabri S. Four aspects affecting health economic decision models and their validation. PharmacoEconomics. 2022;40: 241--248. doi:\href{https://doi.org/10.1007/s40273-021-01110-w}{10.1007/s40273-021-01110-w}}

\leavevmode\vadjust pre{\hypertarget{ref-Arnold2010}{}}%
\CSLLeftMargin{13. }%
\CSLRightInline{Arnold RJG, Ekins S. Time for cooperation in health economics among the modelling community. PharmacoEconomics. 2010;28: 609--613. doi:\href{https://doi.org/10.2165/11537580-000000000-00000}{10.2165/11537580-000000000-00000}}

\leavevmode\vadjust pre{\hypertarget{ref-barbieri2010international}{}}%
\CSLLeftMargin{14. }%
\CSLRightInline{Barbieri M, Drummond M, Rutten F, Cook J, Glick HA, Lis J, et al. What do international pharmacoeconomic guidelines say about economic data transferability? Value in Health. Elsevier; 2010;13: 1028--1037. }

\leavevmode\vadjust pre{\hypertarget{ref-garcia2021cost}{}}%
\CSLLeftMargin{15. }%
\CSLRightInline{Garcı́a-Mochón L, Rovira Forns J, Espin J. Cost transferability problems in economic evaluation as a framework for an european health care and social costs database. Cost Effectiveness and Resource Allocation. Springer; 2021;19: 1--12. }

\leavevmode\vadjust pre{\hypertarget{ref-Jenkins2021}{}}%
\CSLLeftMargin{16. }%
\CSLRightInline{Jenkins DA, Martin GP, Sperrin M, Riley RD, Debray TPA, Collins GS, et al. Continual updating and monitoring of clinical prediction models: Time for dynamic prediction systems? Diagnostic and Prognostic Research. 2021;5: 1. doi:\href{https://doi.org/10.1186/s41512-020-00090-3}{10.1186/s41512-020-00090-3}}

\leavevmode\vadjust pre{\hypertarget{ref-Sampson2017}{}}%
\CSLLeftMargin{17. }%
\CSLRightInline{Sampson CJ, Wrightson T. Model registration: A call to action. PharmacoEconomics - Open. 2017;1: 73--77. doi:\href{https://doi.org/10.1007/s41669-017-0019-2}{10.1007/s41669-017-0019-2}}

\leavevmode\vadjust pre{\hypertarget{ref-pan2021modular}{}}%
\CSLLeftMargin{18. }%
\CSLRightInline{Pan M, Gawthrop PJ, Cursons J, Crampin EJ. Modular assembly of dynamic models in systems biology. PLoS computational biology. Public Library of Science San Francisco, CA USA; 2021;17: e1009513. }

\leavevmode\vadjust pre{\hypertarget{ref-barros2023empowering}{}}%
\CSLLeftMargin{19. }%
\CSLRightInline{Barros C, Luo Y, Chubaty AM, Eddy IM, Micheletti T, Boisvenue C, et al. Empowering ecological modellers with a PERFICT workflow: Seamlessly linking data, parameterisation, prediction, validation and visualisation. Methods in Ecology and Evolution. Wiley Online Library; 2023; }

\leavevmode\vadjust pre{\hypertarget{ref-frazier2022robust}{}}%
\CSLLeftMargin{20. }%
\CSLRightInline{Frazier-Logue N, Wang J, Wang Z, Sodums D, Khosla A, Samson AD, et al. A robust modular automated neuroimaging pipeline for model inputs to TheVirtualBrain. Frontiers in Neuroinformatics. Frontiers Media SA; 2022;16: 883223. }

\leavevmode\vadjust pre{\hypertarget{ref-Afzali2013}{}}%
\CSLLeftMargin{21. }%
\CSLRightInline{Afzali HH, Karnon J, Merlin T. Improving the accuracy and comparability of model-based economic evaluations of health technologies for reimbursement decisions: A methodological framework for the development of reference models. Med Decis Making. 2013;33: 325--32. doi:\href{https://doi.org/10.1177/0272989x12458160}{10.1177/0272989x12458160}}

\leavevmode\vadjust pre{\hypertarget{ref-Trauer2017}{}}%
\CSLLeftMargin{22. }%
\CSLRightInline{Trauer JM, Ragonnet R, Doan TN, McBryde ES. Modular programming for tuberculosis control, the {``AuTuMN''} platform. BMC Infectious Diseases. 2017;17: 546. doi:\href{https://doi.org/10.1186/s12879-017-2648-6}{10.1186/s12879-017-2648-6}}

\leavevmode\vadjust pre{\hypertarget{ref-Urach2013}{}}%
\CSLLeftMargin{23. }%
\CSLRightInline{Urach C, Zauner G, Endel G, Wilbacher I, Breitenecker F. A modular simulation model for assessing interventions for abdominal aortic aneurysms. 2013 winter simulations conference (WSC). 2013. pp. 66--76. doi:\href{https://doi.org/10.1109/WSC.2013.6721408}{10.1109/WSC.2013.6721408}}

\leavevmode\vadjust pre{\hypertarget{ref-Pouwels2022}{}}%
\CSLLeftMargin{24. }%
\CSLRightInline{Pouwels X, Sampson CJ, Arnold RJG. Opportunities and barriers to the development and use of open source health economic models: A survey. Value Health. 2022;25: 473--479. doi:\href{https://doi.org/10.1016/j.jval.2021.10.001}{10.1016/j.jval.2021.10.001}}

\leavevmode\vadjust pre{\hypertarget{ref-Emerson2019}{}}%
\CSLLeftMargin{25. }%
\CSLRightInline{Emerson J, Bacon R, Kent A, Neumann PJ, Cohen JT. Publication of decision model source code: Attitudes of health economics authors. PharmacoEconomics. 2019;37: 1409--1410. doi:\href{https://doi.org/10.1007/s40273-019-00796-3}{10.1007/s40273-019-00796-3}}

\leavevmode\vadjust pre{\hypertarget{ref-Michalczyk2018}{}}%
\CSLLeftMargin{26. }%
\CSLRightInline{Michalczyk J, Clay E, Pochopien M, Aballea S. PRM123 - AN OVERVIEW OF OPEN-SOURCE MODELS IN HEALTH ECONOMICS. Value in Health. 2018;21: S377. doi:\href{https://doi.org/10.1016/j.jval.2018.09.2243}{10.1016/j.jval.2018.09.2243}}

\leavevmode\vadjust pre{\hypertarget{ref-Wu2019}{}}%
\CSLLeftMargin{27. }%
\CSLRightInline{Wu EQ, Zhou Z-Y, Xie J, Metallo C, Thokala P. Transparency in health economic modeling: Options, issues and potential solutions. PharmacoEconomics. 2019;37: 1349--1354. doi:\href{https://doi.org/10.1007/s40273-019-00842-0}{10.1007/s40273-019-00842-0}}

\leavevmode\vadjust pre{\hypertarget{ref-Sampson2019}{}}%
\CSLLeftMargin{28. }%
\CSLRightInline{Sampson CJ, Arnold R, Bryan S, Clarke P, Ekins S, Hatswell A, et al. Transparency in decision modelling: What, why, who and how? PharmacoEconomics. 2019;37: 1355--1369. doi:\href{https://doi.org/10.1007/s40273-019-00819-z}{10.1007/s40273-019-00819-z}}

\leavevmode\vadjust pre{\hypertarget{ref-RN73}{}}%
\CSLLeftMargin{29. }%
\CSLRightInline{Long KM, Meadows GN. Simulation modelling in mental health: A systematic review. Journal of Simulation. 2017; doi:\href{https://doi.org/10.1057/s41273-017-0062-0}{10.1057/s41273-017-0062-0}}

\leavevmode\vadjust pre{\hypertarget{ref-Basu2018}{}}%
\CSLLeftMargin{30. }%
\CSLRightInline{Clearinghouse C. Basu, kim: Alcohol use disorder {[}Internet{]}. OSF; 2018. Available: \href{https://osf.io/jvayu}{osf.io/jvayu}}

\leavevmode\vadjust pre{\hypertarget{ref-OSMC_20xx}{}}%
\CSLLeftMargin{31. }%
\CSLRightInline{Evaluation of Value C for the, Health R in. Open-source model clearinghouse {[}Internet{]}. Tufts Medical Center; Available: \url{http://ghcearegistry.org/orchard/open-source-model-clearinghouse}}

\leavevmode\vadjust pre{\hypertarget{ref-IVIMDD2022}{}}%
\CSLLeftMargin{32. }%
\CSLRightInline{Innovation T, Initiative V. IVI-MDD value model {[}Internet{]}. 2022. Available: \url{https://www.thevalueinitiative.org/ivi-mdd-value-model/}}

\leavevmode\vadjust pre{\hypertarget{ref-Wilson_2017}{}}%
\CSLLeftMargin{33. }%
\CSLRightInline{Wilson JAC Greg AND Bryan. Good enough practices in scientific computing. PLOS Computational Biology. Public Library of Science; 2017;13: 1--20. doi:\href{https://doi.org/10.1371/journal.pcbi.1005510}{10.1371/journal.pcbi.1005510}}

\leavevmode\vadjust pre{\hypertarget{ref-Alarid2019}{}}%
\CSLLeftMargin{34. }%
\CSLRightInline{Alarid-Escudero F, Krijkamp EM, Pechlivanoglou P, Jalal H, Kao S-YZ, Yang A, et al. A need for change! A coding framework for improving transparency in decision modeling. PharmacoEconomics. 2019;37: 1329--1339. doi:\href{https://doi.org/10.1007/s40273-019-00837-x}{10.1007/s40273-019-00837-x}}

\leavevmode\vadjust pre{\hypertarget{ref-8717448}{}}%
\CSLLeftMargin{35. }%
\CSLRightInline{Hourani H, Wasmi H, Alrawashdeh T. A code complexity model of object oriented programming (OOP). 2019 IEEE jordan international joint conference on electrical engineering and information technology (JEEIT). 2019. pp. 560--564. doi:\href{https://doi.org/10.1109/JEEIT.2019.8717448}{10.1109/JEEIT.2019.8717448}}

\leavevmode\vadjust pre{\hypertarget{ref-7181447}{}}%
\CSLLeftMargin{36. }%
\CSLRightInline{Milojkovic N, Caracciolo A, Lungu MF, Nierstrasz O, Röthlisberger D, Robbes R. Polymorphism in the spotlight: Studying its prevalence in java and smalltalk. 2015 IEEE 23rd international conference on program comprehension. 2015. pp. 186--195. doi:\href{https://doi.org/10.1109/ICPC.2015.29}{10.1109/ICPC.2015.29}}

\leavevmode\vadjust pre{\hypertarget{ref-techver2019}{}}%
\CSLLeftMargin{37. }%
\CSLRightInline{Büyükkaramikli NC, Rutten-van Mölken MPMH, Severens JL, Al M. TECH-VER: A verification checklist to reduce errors in models and improve their credibility. PharmacoEconomics. 2019;37: 1391--1408. doi:\href{https://doi.org/10.1007/s40273-019-00844-y}{10.1007/s40273-019-00844-y}}

\leavevmode\vadjust pre{\hypertarget{ref-ERICWONG2010188}{}}%
\CSLLeftMargin{38. }%
\CSLRightInline{Eric Wong W, Debroy V, Choi B. A family of code coverage-based heuristics for effective fault localization. Journal of Systems and Software. 2010;83: 188--208. doi:\url{https://doi.org/10.1016/j.jss.2009.09.037}}

\leavevmode\vadjust pre{\hypertarget{ref-Radeva2020}{}}%
\CSLLeftMargin{39. }%
\CSLRightInline{Radeva D, Hopkin G, Mossialos E, Borrill J, Osipenko L, Naci H. Assessment of technical errors and validation processes in economic models submitted by the company for NICE technology appraisals. International Journal of Technology Assessment in Health Care. 2020;36: 311--316. doi:\href{https://doi.org/10.1017/S0266462320000422}{10.1017/S0266462320000422}}

\leavevmode\vadjust pre{\hypertarget{ref-semver20xx}{}}%
\CSLLeftMargin{40. }%
\CSLRightInline{Preston-Werner T. Semantic versioning 2.0.0 {[}Internet{]}. 2022. Available: \url{https://semver.org}}

\leavevmode\vadjust pre{\hypertarget{ref-CI2017}{}}%
\CSLLeftMargin{41. }%
\CSLRightInline{Shahin M, Ali Babar M, Zhu L. Continuous integration, delivery and deployment: A systematic review on approaches, tools, challenges and practices. IEEE Access. 2017;5: 3909--3943. doi:\href{https://doi.org/10.1109/ACCESS.2017.2685629}{10.1109/ACCESS.2017.2685629}}

\leavevmode\vadjust pre{\hypertarget{ref-ready4oop2022}{}}%
\CSLLeftMargin{42. }%
\CSLRightInline{Hamilton M. Apply an object-oriented paradigm to computational models of mental health systems {[}Internet{]}. 2022. Available: \url{https://ready4-dev.github.io/ready4/articles/V_03.html}}

\leavevmode\vadjust pre{\hypertarget{ref-copyleft2022}{}}%
\CSLLeftMargin{43. }%
\CSLRightInline{Foundation TFS. What is copyleft? {[}Internet{]}. Available: \url{https://www.gnu.org/copyleft/}}

\leavevmode\vadjust pre{\hypertarget{ref-cc02022}{}}%
\CSLLeftMargin{44. }%
\CSLRightInline{Commons C. CC0 1.0 universal {[}Internet{]}. 2022. Available: \url{https://creativecommons.org/publicdomain/zero/1.0/legalcode}}

\leavevmode\vadjust pre{\hypertarget{ref-bysa2022}{}}%
\CSLLeftMargin{45. }%
\CSLRightInline{Commons C. Attribution-ShareAlike 4.0 international {[}Internet{]}. 2022. Available: \url{https://creativecommons.org/licenses/by-sa/4.0/legalcode}}

\leavevmode\vadjust pre{\hypertarget{ref-sampleterms2022}{}}%
\CSLLeftMargin{46. }%
\CSLRightInline{Quantitative Social Science I for. Sample data usage agreement {[}Internet{]}. 2022. Available: \url{https://support.dataverse.harvard.edu/sample-data-usage-agreement}}

\leavevmode\vadjust pre{\hypertarget{ref-Kearns2013}{}}%
\CSLLeftMargin{47. }%
\CSLRightInline{Kearns B, Ara R, Wailoo A, Manca A, Alava MH, Abrams K, et al. Good practice guidelines for the use of statistical regression models in economic evaluations. PharmacoEconomics. 2013;31: 643--652. doi:\href{https://doi.org/10.1007/s40273-013-0069-y}{10.1007/s40273-013-0069-y}}

\leavevmode\vadjust pre{\hypertarget{ref-ready4gh2022}{}}%
\CSLLeftMargin{48. }%
\CSLRightInline{Orygen. ready4: A suite of authoring, modelling and prediction tools for exploring topics in young people's mental health {[}Internet{]}. 2022. Available: \url{https://github.com/ready4-dev/}}

\leavevmode\vadjust pre{\hypertarget{ref-ready4zen2022}{}}%
\CSLLeftMargin{49. }%
\CSLRightInline{Orygen. ready4: Open and modular mental health systems models {[}Internet{]}. 2022. Available: \url{https://zenodo.org/communities/ready4}}

\leavevmode\vadjust pre{\hypertarget{ref-ready4dv2022}{}}%
\CSLLeftMargin{50. }%
\CSLRightInline{Orygen. ready4: Open and modular mental health systems models {[}Internet{]}. 2022. Available: \url{https://dataverse.harvard.edu/dataverse/ready4)}}

\leavevmode\vadjust pre{\hypertarget{ref-ready42022}{}}%
\CSLLeftMargin{51. }%
\CSLRightInline{Hamilton MP. ready4: Implement open science computational models of mental health systems {[}Internet{]}. 2021. doi:\href{https://doi.org/10.5281/zenodo.5606250}{10.5281/zenodo.5606250}}

\leavevmode\vadjust pre{\hypertarget{ref-ready4class2022}{}}%
\CSLLeftMargin{52. }%
\CSLRightInline{Hamilton M, Wiesner G. ready4class: Author Ready4 framework modules {[}Internet{]}. 2022. doi:\href{https://doi.org/10.5281/zenodo.5640313}{10.5281/zenodo.5640313}}

\leavevmode\vadjust pre{\hypertarget{ref-ready4fun2022}{}}%
\CSLLeftMargin{53. }%
\CSLRightInline{Hamilton M, Wiesner G. ready4fun: Author and document functions that extend the Ready4 framework {[}Internet{]}. 2022. doi:\href{https://doi.org/10.5281/zenodo.5611779}{10.5281/zenodo.5611779}}

\leavevmode\vadjust pre{\hypertarget{ref-ready4pack2022}{}}%
\CSLLeftMargin{54. }%
\CSLRightInline{Hamilton M. ready4pack: Author r packages that extend the Ready4 framework {[}Internet{]}. 2022. doi:\href{https://doi.org/10.5281/zenodo.5644322}{10.5281/zenodo.5644322}}

\leavevmode\vadjust pre{\hypertarget{ref-ready4use2022}{}}%
\CSLLeftMargin{55. }%
\CSLRightInline{Hamilton M, Wiesner G. ready4use: Author, label and share Ready4 framework datasets {[}Internet{]}. 2022. doi:\href{https://doi.org/10.5281/zenodo.5644336}{10.5281/zenodo.5644336}}

\leavevmode\vadjust pre{\hypertarget{ref-ready4show2022}{}}%
\CSLLeftMargin{56. }%
\CSLRightInline{Hamilton M, Wiesner G. ready4show: Author literate programs to share insights from applying the Ready4 framework {[}Internet{]}. 2022. doi:\href{https://doi.org/10.5281/zenodo.5644568}{10.5281/zenodo.5644568}}

\leavevmode\vadjust pre{\hypertarget{ref-CRAN2022}{}}%
\CSLLeftMargin{57. }%
\CSLRightInline{Statistical Computing RF for. The comprehensive r archive network {[}Internet{]}. 2022. Available: \url{https://cran.r-project.org}}

\leavevmode\vadjust pre{\hypertarget{ref-hamilton_matthew_2022_6084467}{}}%
\CSLLeftMargin{58. }%
\CSLRightInline{Hamilton M, Gao C. {youthvars: Describe and Validate Youth Mental Health Datasets} {[}Internet{]}. Zenodo; 2022. doi:\href{https://doi.org/10.5281/zenodo.6084467}{10.5281/zenodo.6084467}}

\leavevmode\vadjust pre{\hypertarget{ref-hamilton_matthew_2022_6084824}{}}%
\CSLLeftMargin{59. }%
\CSLRightInline{Hamilton M, Gao C. Scorz: Score questionnaire item responses {[}Internet{]}. Zenodo; 2022. doi:\href{https://doi.org/10.5281/zenodo.6084824}{10.5281/zenodo.6084824}}

\leavevmode\vadjust pre{\hypertarget{ref-hamilton_matthew_2022_6116701}{}}%
\CSLLeftMargin{60. }%
\CSLRightInline{Hamilton M, Gao C. {specific: Specify Candidate Models for Representing Mental Health Systems} {[}Internet{]}. Zenodo; 2022. doi:\href{https://doi.org/10.5281/zenodo.6116701}{10.5281/zenodo.6116701}}

\leavevmode\vadjust pre{\hypertarget{ref-gao_caroline_2022_6130155}{}}%
\CSLLeftMargin{61. }%
\CSLRightInline{Gao C, Hamilton M. {TTU: Implement Transfer to Utility Mapping Algorithms} {[}Internet{]}. Zenodo; 2022. doi:\href{https://doi.org/10.5281/zenodo.6130155}{10.5281/zenodo.6130155}}

\leavevmode\vadjust pre{\hypertarget{ref-matthew_p_hamilton_2021_5646669}{}}%
\CSLLeftMargin{62. }%
\CSLRightInline{Hamilton MP, Gao CX. Youthu: Transform youth outcomes to health utility predictions {[}Internet{]}. Zenodo; 2022. doi:\href{https://doi.org/10.5281/zenodo.6210978}{10.5281/zenodo.6210978}}

\leavevmode\vadjust pre{\hypertarget{ref-hamilton_matthew_2022_6129906}{}}%
\CSLLeftMargin{63. }%
\CSLRightInline{Hamilton M, Gao C. {Complete study program to reproduce all steps from data ingest through to results dissemination for a study to map mental health measures to AQoL-6D health utility} {[}Internet{]}. Zenodo; 2022. doi:\href{https://doi.org/10.5281/zenodo.6212704}{10.5281/zenodo.6212704}}

\leavevmode\vadjust pre{\hypertarget{ref-hamilton_matthew_2022_6416330}{}}%
\CSLLeftMargin{64. }%
\CSLRightInline{Hamilton M, Gao C. {aqol6dmap\_use: Apply AQoL-6D Utility Mapping Models To New Data} {[}Internet{]}. Zenodo; 2022. doi:\href{https://doi.org/10.5281/zenodo.6416330}{10.5281/zenodo.6416330}}

\leavevmode\vadjust pre{\hypertarget{ref-hamilton_matthew_p_2022_6321821}{}}%
\CSLLeftMargin{65. }%
\CSLRightInline{Hamilton MP. {aqol6dmap\_fakes: Generate fake input data for an AQoL-6D mapping study} {[}Internet{]}. Zenodo; 2022. doi:\href{https://doi.org/10.5281/zenodo.6321821}{10.5281/zenodo.6321821}}

\leavevmode\vadjust pre{\hypertarget{ref-hamilton_matthew_2022_6116385}{}}%
\CSLLeftMargin{66. }%
\CSLRightInline{Hamilton M. {ttu\_mdl\_ctlg: Generate a template utility mapping (transfer to utility) model catalogue} {[}Internet{]}. Zenodo; 2022. doi:\href{https://doi.org/10.5281/zenodo.6116385}{10.5281/zenodo.6116385}}

\leavevmode\vadjust pre{\hypertarget{ref-matthew_p_hamilton_2022_5976988}{}}%
\CSLLeftMargin{67. }%
\CSLRightInline{Hamilton MP. {ready4-dev/ttu\_lng\_ss: Create a Draft Scientific Manuscript For A Utility Mapping Study} {[}Internet{]}. Zenodo; 2022. doi:\href{https://doi.org/10.5281/zenodo.5976988}{10.5281/zenodo.5976988}}

\leavevmode\vadjust pre{\hypertarget{ref-DVNux2fDKDIB0_2021}{}}%
\CSLLeftMargin{68. }%
\CSLRightInline{Hamilton MP, Gao CX, Filia KM, Menssink JM, Sharmin S, Telford N, et al. {Transfer to AQoL-6D Utility Mapping Algorithms} {[}Internet{]}. Harvard Dataverse; 2021. doi:\href{https://doi.org/10.7910/DVN/DKDIB0}{10.7910/DVN/DKDIB0}}

\leavevmode\vadjust pre{\hypertarget{ref-DVNux2fHJXYKQ_2021}{}}%
\CSLLeftMargin{69. }%
\CSLRightInline{Hamilton MP. {Synthetic (fake) youth mental health datasets and data dictionaries} {[}Internet{]}. Harvard Dataverse; 2021. doi:\href{https://doi.org/10.7910/DVN/HJXYKQ}{10.7910/DVN/HJXYKQ}}

\leavevmode\vadjust pre{\hypertarget{ref-Hamilton2021.07.07.21260129}{}}%
\CSLLeftMargin{70. }%
\CSLRightInline{Hamilton MP, Gao CX, Filia KM, Menssink JM, Sharmin S, Telford N, et al. Predicting quality adjusted life years in young people attending primary mental health services. medRxiv. Cold Spring Harbor Laboratory Press; 2021; doi:\href{https://doi.org/10.1101/2021.07.07.21260129}{10.1101/2021.07.07.21260129}}

\leavevmode\vadjust pre{\hypertarget{ref-hamilton_matthew_2022_6627995}{}}%
\CSLLeftMargin{71. }%
\CSLRightInline{Hamilton M. {dce\_sa\_design: An R Markdown program to create the experimental design for a Discrete Choice Experiment (DCE) exploring online help seeking in socially anxious young people} {[}Internet{]}. Zenodo; 2022. doi:\href{https://doi.org/10.5281/zenodo.6627995}{10.5281/zenodo.6627995}}

\leavevmode\vadjust pre{\hypertarget{ref-DVNux2fV3OKZV_2022}{}}%
\CSLLeftMargin{72. }%
\CSLRightInline{Hamilton M. {Springtides reports for Local Government Areas in the North West of Melbourne} {[}Internet{]}. Harvard Dataverse; 2022. doi:\href{https://doi.org/10.7910/DVN/V3OKZV}{10.7910/DVN/V3OKZV}}

\leavevmode\vadjust pre{\hypertarget{ref-RN33}{}}%
\CSLLeftMargin{73. }%
\CSLRightInline{Hamilton MP, Hetrick SE, Mihalopoulos C, Baker D, Browne V, Chanen AM, et al. Identifying attributes of care that may improve cost-effectiveness in the youth mental health service system. Med J Aust. 2017;207: S27--S37. doi:\href{https://doi.org/10.5694/mja17.00972}{10.5694/mja17.00972}}

\leavevmode\vadjust pre{\hypertarget{ref-rfwn2022}{}}%
\CSLLeftMargin{74. }%
\CSLRightInline{Orygen. ready4-dev - documenting the development of an open souce youth mental health systems model {[}Internet{]}. Available: \url{https://ready4-dev.com/}}

\leavevmode\vadjust pre{\hypertarget{ref-RN8}{}}%
\CSLLeftMargin{75. }%
\CSLRightInline{Bloom DE, Cafiero ET, Jané-Llopis E, Abrahams-Gessel S, Bloom LR, Fathima S, et al. The global economic burden of noncommunicable diseases. 91-93 route de la Capite,CH-1223 Cologny/Geneva,Switzerland: World Economic Forum.; 2011. }

\leavevmode\vadjust pre{\hypertarget{ref-GBD2019}{}}%
\CSLLeftMargin{76. }%
\CSLRightInline{Global, regional, and national burden of 12 mental disorders in 204 countries and territories, 1990\&\#x2013;2019: A systematic analysis for the global burden of disease study 2019. The Lancet Psychiatry. 2022;9: 137--150. doi:\href{https://doi.org/10.1016/S2215-0366(21)00395-3}{10.1016/S2215-0366(21)00395-3}}

\leavevmode\vadjust pre{\hypertarget{ref-RN25}{}}%
\CSLLeftMargin{77. }%
\CSLRightInline{Chisholm D, Sweeny K, Sheehan P, Rasmussen B, Smit F, Cuijpers P, et al. Scaling-up treatment of depression and anxiety: A global return on investment analysis. The Lancet Psychiatry. 2016; doi:\href{https://doi.org/10.1016/s2215-0366(16)30024-4}{10.1016/s2215-0366(16)30024-4}}

\leavevmode\vadjust pre{\hypertarget{ref-RN22}{}}%
\CSLLeftMargin{78. }%
\CSLRightInline{Saxena S, Thornicroft G, Knapp M, Whiteford H. Resources for mental health: Scarcity, inequity, and inefficiency. The Lancet. 370: 878--889. doi:\href{https://doi.org/10.1016/S0140-6736(07)61239-2}{10.1016/S0140-6736(07)61239-2}}

\leavevmode\vadjust pre{\hypertarget{ref-RN23}{}}%
\CSLLeftMargin{79. }%
\CSLRightInline{Whiteford H, Ferrari A, Degenhardt L. Global burden of disease studies: Implications for mental and substance use disorders. Health Affairs. 2016;35: 1114--1120. doi:\href{https://doi.org/10.1377/hlthaff.2016.0082}{10.1377/hlthaff.2016.0082}}

\leavevmode\vadjust pre{\hypertarget{ref-20211700}{}}%
\CSLLeftMargin{80. }%
\CSLRightInline{Santomauro DF, Mantilla Herrera AM, Shadid J, Zheng P, Ashbaugh C, Pigott DM, et al. Global prevalence and burden of depressive and anxiety disorders in 204 countries and territories in 2020 due to the COVID-19 pandemic. The Lancet. 2021;398: 1700--1712. doi:\url{https://doi.org/10.1016/S0140-6736(21)02143-7}}

\leavevmode\vadjust pre{\hypertarget{ref-page_howard_2010}{}}%
\CSLLeftMargin{81. }%
\CSLRightInline{Page LA, Howard LM. The impact of climate change on mental health (but will mental health be discussed at copenhagen?). Psychological Medicine. Cambridge University Press; 2010;40: 177--180. doi:\href{https://doi.org/10.1017/S0033291709992169}{10.1017/S0033291709992169}}

\leavevmode\vadjust pre{\hypertarget{ref-RN11}{}}%
\CSLLeftMargin{82. }%
\CSLRightInline{Organization WH, Foundation CG. Social determinants of mental health. Geneva: World Health Organization; 2014. }

\leavevmode\vadjust pre{\hypertarget{ref-Fried2020}{}}%
\CSLLeftMargin{83. }%
\CSLRightInline{Fried EI, Robinaugh DJ. Systems all the way down: Embracing complexity in mental health research. BMC Medicine. 2020;18: 205. doi:\href{https://doi.org/10.1186/s12916-020-01668-w}{10.1186/s12916-020-01668-w}}

\leavevmode\vadjust pre{\hypertarget{ref-RN2111}{}}%
\CSLLeftMargin{84. }%
\CSLRightInline{Langellier BA, Yang Y, Purtle J, Nelson KL, Stankov I, Diez Roux AV. Complex systems approaches to understand drivers of mental health and inform mental health policy: A systematic review. Administration And Policy In Mental Health. 2018; doi:\href{https://doi.org/10.1007/s10488-018-0887-5}{10.1007/s10488-018-0887-5}}

\leavevmode\vadjust pre{\hypertarget{ref-RN26}{}}%
\CSLLeftMargin{85. }%
\CSLRightInline{Jorm AF, Patten SB, Brugha TS, Mojtabai R. Has increased provision of treatment reduced the prevalence of common mental disorders? Review of the evidence from four countries. World psychiatry : official journal of the World Psychiatric Association (WPA). 2017;16: 90--99. doi:\href{https://doi.org/10.1002/wps.20388}{10.1002/wps.20388}}

\leavevmode\vadjust pre{\hypertarget{ref-RN42}{}}%
\CSLLeftMargin{86. }%
\CSLRightInline{Furst MA, Gandré C, Romero López-Alberca C, Salvador-Carulla L. Healthcare ecosystems research in mental health: A scoping review of methods to describe the context of local care delivery. BMC Health Services Research. 2019;19: 173. doi:\href{https://doi.org/10.1186/s12913-019-4005-5}{10.1186/s12913-019-4005-5}}

\leavevmode\vadjust pre{\hypertarget{ref-RN43}{}}%
\CSLLeftMargin{87. }%
\CSLRightInline{Alegría M, NeMoyer A, Falgàs Bagué I, Wang Y, Alvarez K. Social determinants of mental health: Where we are and where we need to go. Current Psychiatry Reports. 2018;20: 95--95. doi:\href{https://doi.org/10.1007/s11920-018-0969-9}{10.1007/s11920-018-0969-9}}

\leavevmode\vadjust pre{\hypertarget{ref-nmhsmn2022}{}}%
\CSLLeftMargin{88. }%
\CSLRightInline{Mental Health Research QC for. National mental health systems modelling network {[}Internet{]}. 2022. Available: \url{https://qcmhr.org/research/research-streams/mental-health-services-research/national-mental-health-systems-modelling-network/}}

\leavevmode\vadjust pre{\hypertarget{ref-Zabell2021}{}}%
\CSLLeftMargin{89. }%
\CSLRightInline{Zabell T, Long KM, Scott D, Hope J, McLoughlin I, Enticott J. Engaging healthcare staff and stakeholders in healthcare simulation modeling to better translate research into health impact: A systematic review. Frontiers in Health Services. 2021;1. doi:\href{https://doi.org/10.3389/frhs.2021.644831}{10.3389/frhs.2021.644831}}

\leavevmode\vadjust pre{\hypertarget{ref-SQUIRES2016588}{}}%
\CSLLeftMargin{90. }%
\CSLRightInline{Squires H, Chilcott J, Akehurst R, Burr J, Kelly MP. A framework for developing the structure of public health economic models. Value in Health. 2016;19: 588--601. doi:\url{https://doi.org/10.1016/j.jval.2016.02.011}}

\end{CSLReferences}

\newpage
\appendix
\counterwithin{figure}{section}
\counterwithin{table}{section}

\end{document}
